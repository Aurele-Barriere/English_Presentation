\documentclass{beamer}
\usetheme{Dresden}
\usecolortheme{beaver}

\setbeamertemplate{navigation symbols}{}%remove navigation symbols

\defbeamertemplate*{headline}{miniframes theme no subsection}
{%
  \begin{beamercolorbox}[colsep=1.5pt]{upper separation line head}
  \end{beamercolorbox}
  \begin{beamercolorbox}{section in head/foot}
    \vskip2pt\insertnavigation{\paperwidth}\vskip2pt
  \end{beamercolorbox}%
  \begin{beamercolorbox}[colsep=1.5pt]{lower separation line head}
  \end{beamercolorbox}
}

\setbeamertemplate{footline}[miniframes theme no subsection]


\title{Voting Systems}
\author{Aur\`ele Barri\`ere \& Sol\`ene Mirliaz}
\date{04/04/2017}

\begin{document}

\begin{frame}
\maketitle
\end{frame}

\section{Historical Approach}
\begin{frame}{\secname}
  \begin{itemize}
  \item Bipartite systems have problems
  \item There are many voting systems
  \item People are never happy
  \item We use voting in many different situations
  \end{itemize}
  
\end{frame}

\section{Examples of voting systems}
\begin{frame}{\secname}
  Majorite, duels, elimination...
\end{frame}



\section{Social Choice Theory}
\subsection{A formalism for voting systems}
\begin{frame}{\secname: \subsecname}

\end{frame}

\subsection{Interesting Properties}
\begin{frame}{\secname: \subsecname}

\end{frame}


\subsection{Impossibility Theorems}
\begin{frame}{\secname: \subsecname}

\end{frame}


\section{Towards better voting systems}
\begin{frame}{\secname}

\end{frame}


\end{document}
